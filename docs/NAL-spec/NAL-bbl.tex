\begin{thebibliography}{}

\bibitem[Wang, 1994]{wp:nal2}
Wang, P. (1994).
\newblock From inheritance relation to nonaxiomatic logic.
\newblock {\em International Journal of Approximate Reasoning}, 11(4):281--319.

\bibitem[Wang, 1995a]{wp:phd}
Wang, P. (1995a).
\newblock {\em Non-Axiomatic Reasoning System: Exploring the Essence of
  Intelligence}.
\newblock PhD thesis, Indiana University.

\bibitem[Wang, 1995b]{wp:ref2}
Wang, P. (1995b).
\newblock Reference classes and multiple inheritances.
\newblock {\em International Journal of Uncertainty, Fuzziness and and
  Knowledge-based Systems}, 3(1):79--91.

\bibitem[Wang, 1996a]{wp:bias2}
Wang, P. (1996a).
\newblock Heuristics and normative models of judgment under uncertainty.
\newblock {\em International Journal of Approximate Reasoning}, 14(4):221--235.

\bibitem[Wang, 1996b]{wp:fuzzy2}
Wang, P. (1996b).
\newblock The interpretation of fuzziness.
\newblock {\em IEEE Transactions on Systems, Man, and Cybernetics, Part B:
  Cybernetics}, 26(4):321--326.

\bibitem[Wang, 1996c]{wp:reso2}
Wang, P. (1996c).
\newblock Problem-solving under insufficient resources.
\newblock In {\em Working Notes of the AAAI Fall Symposium on Flexible
  Computation}, pages 148--155, Cambridge, Massachusetts.

\bibitem[Wang, 2000]{wp:syllogism}
Wang, P. (2000).
\newblock Unified inference in extended syllogism.
\newblock In Flach, P. and Kakas, A., editors, {\em Abduction and Induction:
  Essays on Their Relation and Integration}, pages 117--129. Kluwer Academic
  Publishers, Dordrecht.

\bibitem[Wang, 2001a]{wp:abd}
Wang, P. (2001a).
\newblock Abduction in non-axiomatic logic.
\newblock In {\em Working Notes of the IJCAI workshop on Abductive Reasoning},
  pages 56--63, Seattle, Washington.

\bibitem[Wang, 2001b]{wp:higher2}
Wang, P. (2001b).
\newblock Confidence as higher-order uncertainty.
\newblock In {\em Proceedings of the Second International Symposium on
  Imprecise Probabilities and Their Applications}, pages 352--361, Ithaca, New
  York.

\bibitem[Wang, 2004a]{wp:bayes3}
Wang, P. (2004a).
\newblock The limitation of {B}ayesianism.
\newblock {\em Artificial Intelligence}, 158(1):97--106.

\bibitem[Wang, 2004b]{wp:reso3}
Wang, P. (2004b).
\newblock Problem solving with insufficient resources.
\newblock {\em International Journal of Uncertainty, Fuzziness and and
  Knowledge-based Systems}, 12(5):673--700.

\bibitem[Wang, 2004c]{wp:unify}
Wang, P. (2004c).
\newblock Toward a unified artificial intelligence.
\newblock In {\em Papers from the 2004 AAAI Fall Symposium on Achieving
  Human-Level Intelligence through Integrated Research and Systems}, pages
  83--90, Washington DC.

\bibitem[Wang, 2005]{wp:seman2}
Wang, P. (2005).
\newblock Experience-grounded semantics: a theory for intelligent systems.
\newblock {\em Cognitive Systems Research}, 6(4):282--302.

\bibitem[Wang, 2006]{wp:book1}
Wang, P. (2006).
\newblock {\em Rigid Flexibility: The Logic of Intelligence}.
\newblock Springer, Dordrecht.

\bibitem[Wang, 2007a]{wp:roadmap}
Wang, P. (2007a).
\newblock From {NARS} to a thinking machine.
\newblock In Goertzel, B. and Wang, P., editors, {\em Advance of Artificial
  General Intelligence}, pages 75--93. IOS Press, Amsterdam.

\bibitem[Wang, 2007b]{wp:agi}
Wang, P. (2007b).
\newblock The logic of intelligence.
\newblock In Goertzel, B. and Pennachin, C., editors, {\em Artificial General
  Intelligence}, pages 31--62. Springer, Berlin.

\bibitem[Wang, 2009a]{wp:analogy}
Wang, P. (2009a).
\newblock Analogy in a general-purpose reasoning system.
\newblock {\em Cognitive Systems Research}, 10(3):286�--296.

\bibitem[Wang, 2009b]{wp:CBC}
Wang, P. (2009b).
\newblock Case-by-case problem solving.
\newblock In {\em Proceedings of the Second Conference on Artificial General
  Intelligence}, pages 180--185.

\bibitem[Wang, 2009c]{wp:formal-evidence}
Wang, P. (2009c).
\newblock Formalization of evidence: A comparative study.
\newblock {\em Journal of Artificial General Intelligence}, 1:25--53.

\end{thebibliography}
